\documentclass[12pt]{article}
\usepackage{amssymb}
\usepackage{amsmath}
\usepackage{float}

%BEGIN for \sha (dirac comb)
\usepackage[OT2,T1]{fontenc}
\DeclareSymbolFont{cyrletters}{OT2}{wncyr}{m}{n}
\DeclareMathSymbol{\sha}{\mathalpha}{cyrletters}{"58}
%END for \sha (dirac comb)

\title{MUMT605 Assignment 1}
\author{Johnty Wang}
\date{13 Oct 2014}

\begin{document}
\maketitle
\section*{Part 1}
For this part, I implemented a matlab function called \verb|A1_func| in Matlab, with its help/description as follows:

\begin{verbatim}
% A1_func generates a rectangular wave with specified parameters
% W = A1_func(f, d_cycle, ph, time, sample_rate, doPlot)
%   - f frequency in hertz
%   - d_cycle duty cycle of wave (0.0 - 1.0)
%   - ph phase offset (as portion of period, 0.0 - 1.0)
%   - time duration of output wave in seconds
%   - sample_rate output sample rate
%   - doPlot 1=plot spectrum, wave, etc 0 = no plot
%   - W the output samples
\end{verbatim}

The internal code comments provide explanation of the process. Specifically, after I obtained the initial frequency-independent wavetable, I made two attempts at interpolating towards the final target frequency:

\begin{enumerate}
\item {\textit{The naïve approach}: for the first attempt, I thought that by simply adjusting the duration of the final waveform, I could get different frequencies, like follows:

The frequency of a given wavetable, at a certain sample rate is:
\begin{equation}
f_0 = \frac{F_s}{N}
\end{equation}
To obtain a desired frequency, we can simply go:
\begin{equation}\label{eq:2}
f_{\text{target}} = \frac{F_s}{N_{\text{target}}}
\end{equation}
Which means the desired table size of the new wavetable should be:
\begin{equation}
N_{\text{target}} = \frac{F_s}{f_{\text{target}}}
\end{equation}

However, after I did this, it was clear it had a serious drawback. Since we're working in the discrete time domain, $N_{\text{target}}$ must be integer values. This means that the possible output frequencies are limited to countable numbers of the form in Eq. \ref{eq:2}. In terms of achievable frequencies, we end up with the following examples:

For $F_s$ of 44100 Hz:
\begin{table}[H]\centering
\begin{tabular}{ll}
\hline
$N_{\text{target}}$	& $F$ \\
\hline
100 & 441\\
99	& 445.45\\
98	& 450\\
97	& 454.64\\
50	& 882\\
49	& 900\\
48	& 918\\
\hline
\end{tabular}
\end{table}

Which is clearly inadequate for any kind of musical synthesis!}


\item {\textit{Corrected approach}: The issue in the above approach is due to the limitation of interpolating by modifying the period of the wavetable. I ended up going with the phase increment approach, which adjusts the rate at which the wavetable is read according to the desired frequency. This rate can be adjusted using a continuous variable, and the interpolation is done not by resizing the original table, but instead by looking up at non-discrete locations of the original table. For a thorough implementation more advanced techniques for interpolation should be used, but for my first implementation I simply read the nearest index.}

\end{enumerate}

\subsection*{Putting it together}
Below is a file listing of the submitted assignment:
\begin{itemize}
\item \verb|A1_func|: the main synthesizer code that generates the waveform.
\item \verb|runme.m|: the tester application that does the following:
\begin{enumerate}
	\item 	Runs \verb|A1_func| once with plotting enabled, to see what the output waveform looks like
	\item Calls additional helper functions that generates and plays back a small ``song".
\end{enumerate}
\item \verb|loadscore.m|: a ``song generator" that takes in an input tempo, sample rate, and outputs a hard-coded score made up of frequencies and durations.
\item \verb|note2freq.m|: a quick formula for converting between MIDI notes and frequency.
\end{itemize}

\section*{Part 2}

1.) we know the following:
\begin{itemize}
\item Given
\begin{align*}
y(t) &= \left\lbrace\begin{array}{ll}
x(t) & t\geq 0\\
0 & \text{elsewhere}
\end{array}\right.
\end{align*}
Then,
\begin{align*}
y(|t|) &= \left\lbrace\begin{array}{ll}
x(t) & t\geq 0\\
x(-t) & t < 0
\end{array}\right. \\
\end{align*} 
so for this particular example, we have:
\end{itemize}

\begin{align*}
x(t) &= s(t) + s(-t) \\
\text{from the property of the fourier transform:} \\
\text{if } a(t) \rightarrow A(f) \\
\text{then } a(-t) \rightarrow A(-f) \\
\implies X(f) &= S(f) + S(-f)\\
\implies  X(f) &= \dfrac{1}{\alpha + 2\pi j f} + \dfrac{1}{\alpha + 2\pi j f}\\
&=  \dfrac{\alpha + 2\pi j f + \alpha - 2\pi j f}{(\alpha - 2\pi j f)(\alpha + 2\pi j f)}\\
&= \dfrac{2\alpha}{\alpha^2 + 4\pi f^2}
\end{align*}

2.) \begin{align*}
\sha_{T_0}(t) = \dfrac{1}{T_0}\sum_{n=-\infty}^{\infty} \delta(t - n{T_0}) \\
x_p(t) = x(t) \ast \sha_{T_0} \\
x_p(t) = \dfrac{1}{T_0}\sum_{k=-\infty}^{\infty}e^{2\pi i k \frac{t}{T_0}}x(t) \\
\end{align*}

3.)

4.)


\end{document}